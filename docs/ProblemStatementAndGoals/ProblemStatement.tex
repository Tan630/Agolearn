\documentclass{article}

\usepackage{tabularx}
\usepackage{booktabs}
\usepackage{hyperref}
\usepackage{graphicx}
\usepackage{svg}


\title{Problem Statement and Goals\\\progname}

\author{\authname}

\date{}

%% Comments

\usepackage{color}

\newif\ifcomments\commentstrue %displays comments
%\newif\ifcomments\commentsfalse %so that comments do not display

\ifcomments
\newcommand{\authornote}[3]{\textcolor{#1}{[#3 ---#2]}}
\newcommand{\todo}[1]{\textcolor{red}{[TODO: #1]}}
\newcommand{\note}[1]{\textcolor{red}{[#1 \#NOTE]}}
\else
\newcommand{\authornote}[3]{}
\newcommand{\todo}[1]{}
\fi

\newcommand{\wss}[1]{\authornote{blue}{SS}{#1}} 
\newcommand{\plt}[1]{\authornote{magenta}{TPLT}{#1}} %For explanation of the template
\newcommand{\an}[1]{\authornote{cyan}{Author}{#1}}


%% Common Parts


\newcommand{\progname}{Agolearn} % PUT YOUR PROGRAM NAME HERE
\newcommand{\thisproject}{\progname}
\newcommand{\authname}{Team 1, Agonaught(s)
\\ Yiding Li} % AUTHOR NAMES                  

\usepackage{hyperref}
    \hypersetup{colorlinks=true, linkcolor=blue, citecolor=blue, filecolor=blue,
                urlcolor=blue, unicode=false}
    \urlstyle{same}
                                


\begin{document}

\maketitle

\begin{table}[hp]
\caption{Revision History} \label{TblRevisionHistory}
\begin{tabularx}{\textwidth}{llX}
\toprule
\textbf{Date} & \textbf{Developer(s)} & \textbf{Change}\\
\midrule
2024-01-15 & Yiding Li & First Draft\\
\bottomrule
\end{tabularx}
\end{table}

\section{Problem Statement}

% \wss{You should check your problem statement with the
% \href{https://github.com/smiths/capTemplate/blob/main/docs/Checklists/ProbState-Checklist.pdf}
% {problem statement checklist}.}
% \wss{You can change the section headings, as long as you include the required information.}

\subsection{Motivation}


\hyperref[sec:evalg]{Evolutionary algorithms} lend well to solving novel problems that are difficult to model or solve by traditional means. \hyperref[sec:genalg]{Genetic programming algorithms}, in particular, are able to evolve programs that perform well against arbitrary objective functions.

While evolutionary methods lack the efficiency of traditonal optimization methods (such as simplex algorithms and gradient desgent algorithms), they work in more complex environments (such as moving constraints, moving optimization functions, partially observable environments, and objective functions that do not have well-defined derivatives). This project seeks to develop an optimizer that takes advantage of these features.


\begin{figure}[h]
    \caption{Input and output of a genetic programming algorithm}
    \includegraphics[width=\textwidth]{example\_learn.png}
    \centering
\end{figure}

\subsection{Inputs and Outputs}

Input: 
\textcolor{red}{TODO: Differentiate inputs and outputs for different objectives; if possible, add illustrations.}
\begin{enumerate}
    \item An objective function of either values (vectors of numbers) or functions of real numbers
    \item Specifications for genetic programs, such as (a) node functions, (b) tree depth, and (c) node count
    \item Parameters that control how the algorithm operates, such as (a) choices of evolutoinary operators, (b) number of episodes, and (c) the length of each episode or truncation conditions
\end{enumerate}


Output: A value or genetic program that optimizes the given objective function


\subsection{Stakeholders}
\textcolor{red}{TODO: Enrich this section.}
Computer scientistst that seek to optimize an objective function.

\subsection{Environment}

\textcolor{red}{TODO: Seek advise and improve this section.}
A computer.

\section{Goals}

\textcolor{red}{TODO: Enrich this section. Also, consider moving some from Strech Goals to Goals.}
\begin{itemize}
    \item Optimize against an objective function of values (vectors of real numbers)
\end{itemize}

\section{Stretch Goals}

\begin{itemize}
    \item Optimize against an objective function of functions (genetic programs)
    \item Implement multi-processing to speed up computation (in reference to frameworks such as \href{https://deap.readthedocs.io/en/master/}{Deap})
\end{itemize}
\section{Appendix}

\subsection{Evolutionary Algorithms}
\label{sec:evalg}
\textbf{Evolutionary algorithms} (\textbf{EA}) are optimization algorithms that draw on the evolutionary process. An EA begins with an initial population, then iteratively improves the population through generations by applying various \hyperref[sec:evop]{evolutionary operators}.

\subsection{Evolutionary Operators}
\label{sec:evop}
\textbf{Evolutionary operators} divide into \textbf{parent selectors}, \textbf{variators}, and \textbf{survivor selectors}. These operators emulate events in an evolutionary process:

\begin{itemize}
    \item \textbf{Parent selectors} select from the population to form the parent pool.
    \item \textbf{Variators} act on the parent pool to produce a pool of offsprings. An offspring may inherit traits from parents (by crossover) or posess novel traits (by mutation).
    \item \textbf{Survivor selectors} select from the offspring pool to form the population for the next generation.
\end{itemize}

\begin{figure}[h]
    \textcolor{red}{TODO: Figure is wrong and so is the label.}
    \caption{Evolutionary operators in}
    \includegraphics[width=\textwidth]{complete\_picture}    
    \centering
\end{figure}

\subsection{Genetic Algorithms}
\label{sec:genalg}
\textbf{Genetic algorithms} (\textbf{GA}) are evolutionary algorithms that work with programs. That is, such algorithms evolve \textit{functions} against an objective function that receive functions. Genetic algorithms can evolve agents that behave well in a particular environment (e.g. \href{https://www.mdpi.com/2227-7390/11/13/2931}{a bipedal walker}) or construct mathematical models (e.g. \href{https://link.springer.com/article/10.1007/s11831-023-09922-z}{symbolic regression})

\textcolor{red}{TODO: Tenative: change hyperlinks to numbered references.}



\end{document}